\documentclass{article}
\usepackage[utf8]{inputenc}
\usepackage{gensymb}

\setlength{\parindent}{0pt}

\title{GNSS: Single Point Position Estimation \\
	   \normalsize{TTK5 Lab 1}}
\author{Andreas Nordby Vibeto}
\date{October 2016}

\begin{document}

\maketitle

\section*{Task 1}

When measuring movement and position on earth, two separate measurements is needed. For measuring movement Inertial Measurement Unit (IMU) systems are usially used. These units are attached to the object being measured, and measures all forces "affecting" the object in the Newton law of motion way. When measuring postion, Global Navigation Satelite Systems (GNSS) is the most common way. With GNSS satelites orbiting the earth is used to measure position, and the distances between the object and the satelites are so big that the Newton's law of motion no longer apply. In order to combine these two measurements different coordinate systems is used. *IS THIS PARAGRAPH REALLY NECESSARY?*\\

East North Up (ENU) is a reference frame fixed to the earth surface. Its x-axis points east, y-axis points north and its z-axis points upwards perpendicular to the earth surface. The orign of the fram is on the earth surface, and it uses a reference ellipsiode to describe the earth's curvature.\\

The Earth Centered Earth Fixed (ECEF) reference frame does not have its origin on the earth surface. The origin is in the center of the earth where the x-axis points to the intersection of the 0$\degree$ longitude and latitude, z-axis points to the geographic north along the rotation axis and the y-axis is palced perpendicular to the x- and z-axis. The frame rotates with the earth. When using this frame to describe a postion on the earth surface (or relatively close) longitudenal and latitudinal coordinates are used together with height. For describing objects further away from the surface, like satelites, cartesian coordinates are used.
























\end{document}